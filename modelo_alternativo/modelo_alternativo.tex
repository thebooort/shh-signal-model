\chapter{Modelo alternativo}

\label{ch:modelo_alternativo}

\section{Introducción}

Como adelantábamos en la sección 1.3 si nos centramos en el modelado BEWARE de la transcripción génica esencialmente tenemos dos enfoques: modelar la expresión génica como una cantidad proporcional a la suma ponderada de los factores de transcripción o hacerlo proporcional a la probabilidad de unión del ARN polimerasa, que va modificada por los factores de transcripción.

Tras extenso estudio que se ha hecho del modelo \cite{schaffer}, la multiestabilidad está más que demostrada para conjuntos de valores de parámetros esperables, así como para un conjunto de condiciones iniciales que tambien podrían esperarse en situaciones biologicas. 

Sin embargo, debido a esto (la multiestabilidad que se pone de manifiesto en el estudio del modelo) sabemos que este modelo debe ser modificado, puesto no tiene un soporte experimental biológico contrastado.

Es decir, en experimentos biológicos no observamos este tipo de multiestabilidad. Por tanto si queremos que nuestro modelo de transcripción represente fielmente el comportamiento de la transcripción genetica debemos modificarlo o cambiarlo.

Así, antes de cambiar grandes características de nuestro modelo, nace la idea de modificar nuestra forma de crear el operador BEWARE usando el segundo enfoque que tiene en cuenta el ARN polimerasa. 

Tras el esquema de modelado, vamos a presentar los experimentos numéricos que nos llevan a pensar que podemos estar ante un comportamiento más fiel a lo observado en biología con sólo un estado estable. 

\section{Modelado BEWARE}
Comenzamos aplicando las ideas del método termodinámico a nuestros dos genes, gli y ptc, los cuales suponemos controlados por tres factores de transcripción $\{Gli, Gli_3, Gli3R\} $ que son activador, activador y represor, respectivamente.

\subsubsection{Consideraciones inciales}
\begin{itemize}

\item Destacamos de nuevo, antes de comenzar el resto del calculo del operador, que la expresión de la evolución de la cantidad de las proteínas generadas por gli y ptc, Gli y Ptc, vendrá disminuida por la degradación natural de estas moléculas con dos constantes de degradación.

\item En el modelo, además, las reacciones de unión de los factores de transcripccón y del ARN polimerasa son mucho más rápidos que la síntesis de la proteína Gli o Ptc, por lo tanto, consideramos en equilibrio termodinámico dado por la Ley de Acción de Masas.

\item Por otra parte en este trabajo se considera una versión del operador con cooperatividad total/ausencia de cooperatividad entre factores de transcripción. Podríamos estudiar cómo afecta la hipótesis de tener una cooperatividad parcial en futuros desarrollos del mismo, siguiendo las indicaciones de \cite{cambon1}.
	 
\end{itemize}
Todos los pasos siguen el esquema general presentado en \cite{cambon1}:
\subsection{Calculo del operador}
\subsubsection{Espacio de todas las configuraciones}
Todas las formas posibles de obtener una concentración de equilibrio con $j_{Gli}, j_{Gli_3}, j_{Gli3R},j_P$ activadores, represores y ARN polimerasas son dados por :
\begin{equation}
\begin{split}
&Z^{(3)}(j_{Gli}, j_{Gli_3}, j_{Gli3R},j_P=1;C)=\\&=\textit{C}(C)\frac{n!}{j_{Gli}! j_{Gli_3}! j_{Gli3R}!j_0!}[B]\frac{[RNAP]}{K_{RP}}
(\frac{a_{gli}[Gli]}{K_{Gli}})^{j_{gli}}
(\frac{a_{gli_3}[Gli_3]}{K_{Gli_3}})^{j_{gli_3}}
(\frac{r_{gli3R}[Gli3R]}{K_{Gli3R}})^{j_{gli3R}}
\end{split}
\end{equation}
\begin{equation}
\begin{split}
&Z^{(3)}(j_{Gli}, j_{Gli_3}, j_{Gli3R},j_P=0;C)=\\&=\textit{C}(C)\frac{n!}{j_{Gli}! j_{Gli_3}! j_{Gli3R}!j_0!}[B]\frac{[RNAP]}{K_{RP}}
(\frac{[Gli]}{K_{Gli}})^{j_{gli}}
(\frac{[Gli_3]}{K_{Gli_3}})^{j_{gli_3}}
(\frac{[Gli3R]}{K_{Gli3R}})^{j_{gli3R}}
\end{split}
\end{equation}

donde $j_P = 1$ significa que hay una ARN polimerasa unida y $j_P = 0$ no hay ninguna. Por otra parte, por definición :
$j_0 = n - j_{Gli}- j_{Gli_3}- j_{Gli3R} \geq 0$, y la variable C describe la relación de cooperatividad entre los factores de transcripción. Específicamente, al usar (3) y (4), la función de cooperatividad C toma los valores

\begin{equation}
C(C=\{Gli,Gli_3,Gli3R\}_c)=c^{(j_{Gli}+ j_{Gli_3}+ j_{Gli3R}-1)_+}
\end{equation}
y
\begin{equation}
C(C=\{Gli,Gli_3,Gli3R\}_c)=c^{(j_{Gli}+ j_{Gli_3}+ j_{Gli3R}-1)_+}
\end{equation}

Esto nos permite describir todo el espacio muestral, es decir, el espacio de todas las posibles configuraciones, como:
\begin{equation}
\Omega=\{(j_{Gli}, j_{Gli_3}, j_{Gli3R},j_P);j_{Gli}, j_{Gli_3}, j_{Gli3R}\geq0;j_{Gli}+ j_{Gli_3}+ j_{Gli3R}\leq n,j_p=0,1\}
\end{equation}


\subsubsection{Definicion de la probabilidad de cada configuración}

Una vez que hemos descrito todas las configuraciones posibles en términos de las concentraciones de activador Gli y Gli3, del represor Gli3R y ARN polimerasa, obtenemos fácilmente la probabilidad de encontrar el promotor en una configuración particular de $j_P$ ARN polimerasa y de factores de transcripcion $j_{Gli}, j_{Gli_3}, j_{Gli3R}$ relacionados por una cooperatividad $c$.

\begin{equation}
P^{(3)}(j_{Gli}, j_{Gli_3}, j_{Gli3R},j_P;C)=\frac{Z^{(3)}(j_{Gli}, j_{Gli_3}, j_{Gli3R},j_P;C)}{\sum_{\{j'_{Gli}, j'_{Gli_3}, j'_{Gli3R},j'_P\}\in\Omega}Z^{(3)}(j'_{Gli}, j'_{Gli_3}, j'_{Gli3R},j'_P;C)}
\label{probabilidad}
\end{equation}
 con $(j_{Gli}, j_{Gli_3}, j_{Gli3R},j_P)\in\Omega$
\subsubsection{Resultado final del operador BEWARE}

\begin{equation}
\begin{split}
&BEWARE([Gli][Gli_3][Gli3R][ARNP];C)=\\&=C_B\sum_{j_{Gli}, j_{Gli_3}, j_{Gli3R}\geq0}^{j_{Gli}+ j_{Gli_3}+ j_{Gli3R}\leq n}P^{(n)}(j_{Gli}, j_{Gli_3}, j_{Gli3R},j_P=1;C)
\end{split}
\end{equation}

Al dividir el denominador en dos sumas, segun la ARN polimerasa este ligada o no a la configuración, esta expresión puede reescribirse en términos de la función del factor de regulación, $F_reg$:
\begin{equation}
a
\end{equation}

Realizando los calculos indicados en \cite{cambon1}, vía el paquete \cite{sympy}, este factor de regulación se puede reducir para facilitar el
comprensión del proceso general. Se obtiene así, la expresion:
\begin{equation}
a
\end{equation}

donde $S^{(3)}(x,y,z;C)$ viene dad por la cooperatividad total que hemos asumido:
\begin{equation}
a
\end{equation}

Concluimos así el calculo del modelado del operador BEWARE con un enfoque estimulated. Con este operador y las ecuaciones definidas en el capítulo 1 podemos disponer ya de nuestro modelo completo.

\section{Sistema final}

La mayoría de cuentas del apartado se han generado con la ayuda de \cite{sympy}.

\begin{equation}
\frac{dGli}{dt} = BEWARE(Gli, Gli_3, Gli3R)-k_{deg}Gli,
\label{eq:1}
\end{equation}

\begin{equation}
\frac{dGli_3}{dt} = \frac{r_{g3b}}{Ptc}-Gli_3\left(k_{deg}+\frac{k_{g3rc}}{K_{g3rc}+Signal}\right),
\label{eq:2}
\end{equation}

\begin{equation}
\frac{dGli3R}{dt}= Gli_3\left(\frac{k_{g3rc}}{K_{g3rc}+Signal}\right)-k_{deg}Gli3R,
\label{eq:3}
\end{equation}

\begin{equation}
\frac{dPtc}{dt} = BEWARE(Gli, Gli_3, Gli3R)-k_{degp}Ptc.
\label{eq:4}
\end{equation}


Donde tenemos, por definición:
 \begin{equation}
Signal=\frac{\frac{Shh}{k_{shh}} + 1}{\frac{Shh}{k_{shh}} + 1 + \frac{Ptc}{k_{ptc}}},
\label{signal} \end{equation}

y,


\begin{equation}
BEWARE(Gli, Gli_3, Gli3R)=\frac{c_{b}}{1 + \frac{k_{RNAP}}{F_{reg}(Gli, Gli_3, Gli3R) RNAP}},
\end{equation}

donde solo nos queda describir $F_{reg}$. En el caso de de gradientes opuestos y no/total cooperatividad de los factores de transcripción nos queda:

\begin{equation}
F_{reg}=\frac{1 + \frac{1}{c} \left(\frac{Gli a_{Gli}}{k_{Gli}} c + \frac{Gli_{3} a_{Gli3}}{k_{Gli3R}} c + \frac{Gli3R c}{k_{Gli3R}} r_{Gli3R} + 1\right)^{3} - \frac{1}{c}}{1 + \frac{1}{c} \left(\frac{Gli c}{k_{Gli}} + \frac{Gli_{3} c}{k_{Gli3R}} + \frac{Gli3R c}{k_{Gli3R}} + 1\right)^{3} - \frac{1}{c}}
\end{equation}






Podemos desarrollar las funciones en cada uno de los términos, quedándonos las siguientes expresiones:
\begin{equation}
\frac{dGli}{dt}=- Gli k_{deg} + \frac{c_{b}}{1 + \frac{k_{RNAP} \left(1 + \frac{1}{c} \left(\frac{Gli c}{k_{Gli}} + \frac{Gli_{3} c}{k_{Gli3R}} + \frac{Gli3R c}{k_{Gli3R}} + 1\right)^{3} - \frac{1}{c}\right)}{RNAP \left(1 + \frac{1}{c} \left(\frac{Gli a_{Gli}}{k_{Gli}} c + \frac{Gli_{3} a_{Gli3}}{k_{Gli3R}} c + \frac{Gli3R c}{k_{Gli3R}} r_{Gli3R} + 1\right)^{3} - \frac{1}{c}\right)}}.
\end{equation}


\begin{equation}
\frac{dGli_3}{dt}=- Gli_{3} \left(k_{deg} + \frac{k_{g3rc}}{K_{g3rc} + \frac{\frac{Shh}{k_{shh}} + 1}{\frac{Shh}{k_{shh}} + 1 + \frac{ptc}{k_{ptc}}}}\right) + \frac{r_{g3b}}{ptc}.
\end{equation}

\begin{equation}
\frac{dGli3R}{dt}=Gli_{3} \left(- Gli3R k_{deg} + \frac{k_{g3rc}}{K_{g3rc} + \frac{\frac{Shh}{k_{shh}} + 1}{\frac{Shh}{k_{shh}} + 1 + \frac{ptc}{k_{ptc}}}}\right).
\end{equation}

\begin{equation}
\frac{dPtc}{dt}=\frac{c_{b}}{1 + \frac{k_{RNAP} \left(1 + \frac{1}{c} \left(\frac{Gli c}{k_{Gli}} + \frac{Gli_{3} c}{k_{Gli3R}} + \frac{Gli3R c}{k_{Gli3R}} + 1\right)^{3} - \frac{1}{c}\right)}{RNAP \left(1 + \frac{1}{c} \left(\frac{Gli a_{Gli}}{k_{Gli}} c + \frac{Gli_{3} a_{Gli3}}{k_{Gli3R}} c + \frac{Gli3R c}{k_{Gli3R}} r_{Gli3R} + 1\right)^{3} - \frac{1}{c}\right)}} - k_{deg p} Ptc.
\end{equation}




\section{Estados estacionarios}
Siguiendo con el estudio estandar que se lleva a cabo en los modelos matemáticos procedemos con un estudio sobre los estados estacionarios que podemos encontrar en nuestro modelo.

Frente a los modelos poropuestos anteriormente en \cite{saha,schaffer} nos interesa la posibilidad de no encontrar un interruptor biestable en el comportamiento cualitativo de nuestro modelo. En primer lougar procedemos aforntando el problema desde una perspectiva analítica. 

Sean las ecuaciones \ref{eq:1}\ref{eq:2}\ref{eq:3}\ref{eq:4}, si suponemos que éstas se encuentran en un estado estacionario entonces sus cantidades son constantes. Esto implica que su derivada temporal es igual a cero.

Dado que las ecuaciones continen términos complejos, nos interesamos por agruparlas, de manera que los cálculos no sean más sencillo en un primer intento de extraer información:

Por un lado de \ref{eq:1} y \ref{eq:4}:

$$\begin{cases} 0 = BEWARE(Gli, Gli_3, Gli3R)-k_{deg}Gli, \\0= BEWARE(Gli, Gli_3, Gli3R)-k_{degp}Ptc. \end{cases}$$
Si igualamos ambas ecuaciones nos queda:
\begin{equation}
 k_{deg}Gli=k_{degp}Ptc \implies \frac{k_{deg}}{k_{degp}}Gli=Ptc
\end{equation}

Por otra parte, de \ref{eq:2} y \ref{eq:3}:



$$\begin{cases} 0 = \frac{r_{g3b}}{Ptc}-Gli_3\left(k_{deg}+\frac{k_{g3rc}}{K_{g3rc}+Signal}\right), \\0=Gli_3\left(\frac{k_{g3rc}}{K_{g3rc}+Signal}\right)-k_{deg}Gli3R. \end{cases}$$

Sumando, obtenemos:

\begin{equation}
\begin{split}
0=\frac{r_{g3b}}{Ptc}-Gli_3k_{deg}-k_{deg}Gli3R & \implies \frac{r_{g3b}}{Ptc}=Gli_3k_{deg}+k_{deg}Gli3R\implies
 \\
& \implies \frac{r_{g3b}}{Gli_3k_{deg}+k_{deg}Gli3R}=Ptc
\end{split}
\end{equation}

Con estas cuentas, podemos obtener una función de $Signal$\ref{signal} modificada, la llamaremos $Signal_{modificada}$:
 \begin{equation}
 Signal_{modificada}=\frac{\frac{Shh}{k_{shh}} + 1}{\frac{Shh}{k_{shh}} + 1 + \frac{r_{g3b}}{k_{ptc}(Gli_3k_{deg}+k_{deg}Gli3R)}}.
 \end{equation}
 
Ahora, sustituimos los valores que tenemos para intentar hallar los estados estacionarios. Haciéndolo, \ref{eq:2} nos quedaría:

\begin{equation}
0 = Gli_3k_{deg}+k_{deg}Gli3R-Gli_3\left(k_{deg}+\frac{k_{g3rc}}{K_{g3rc}+Signal_{modificado}}\right),
\label{eq:2-modified}
\end{equation}
y \ref{eq:3}:

\begin{equation}
0=Gli_3\left(\frac{k_{g3rc}}{K_{g3rc}}+Signal_{modificada}\right)-k_{deg}Gli3R.
	\label{eq:3-modified}
\end{equation}

Esto nos deja un sistema de dos ecucaciones con dos incógnitas que resolvemos:




\section{Simulaciones}

\subsection{Variación del operador BEWARE}

\begin{figure}[h]
	\includegraphics[width=0.8\textwidth]{variacion_new_beware}
	\centering
	\caption{Variación del nuevo operador BEWARE }
	\label{vari_beware}
\end{figure}

\begin{figure}[h]
	\includegraphics[width=0.8\textwidth]{variacion_new_beware_2}
	\centering
	\caption{Variación del nuevo operador BEWARE en más rango}
	\label{vari_beware_2}
\end{figure}

\subsection{Evolución temporal}

\begin{figure}[h]
	\includegraphics[width=0.8\textwidth]{new_beware_global}
	\centering
	\caption{Evolución temporal del nuevo operador BEWARE}
	\label{evolu_beware}
\end{figure}

\subsection{Análisis numérico de los estados estacionarios}


