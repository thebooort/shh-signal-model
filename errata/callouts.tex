\documentclass[a4paper,11pt]{article}
\usepackage[spanish]{babel}
\usepackage[utf8]{inputenc}
\usepackage{geometry}

\usepackage{booktabs}  
\usepackage{graphicx} 
\usepackage{listings}
\lstset{%
backgroundcolor=\color{cyan!10},
basicstyle=\ttfamily,
numbers=left,numberstyle=\scriptsize
}

\usepackage[wby]{callouts}

\title{\textbf{Corrigendum} \\ \large{\textit{ Un nuevo modelo matemático para el sistema de señalización de Sonic Hedgehog}}}
\author{Bartolomé Ortiz Viso }

\begin{document}

\maketitle

\section{Introducción}
\begin{itemize}
	\item \textit{Página 1:} \textit{... tejido de brotes \textbf{de miembro}.} 
	
	En referencia a las numerosas situaciones en las que el Shh desempeña un papel fundamental durante el desarrollo.
	\item \textit{Página 8:} \textit{... y un parámetro de saturación $\textbf{K}_{\textbf{g3rc}}$}...
	
	En referencia a la dinámica de Gli3 y Gli3R, error de Latex. 
\end{itemize}
\section{Modelo clásico}
\begin{itemize}
	\item \textit{Página 19:} \textit{... una función \textit{Signal}(\textbf{1.9}) modificada...} 
	
	En referencia al método utilizado para byuscar los valores estacionarios.
\end{itemize}
\section{Modelo alternativo}
	\begin{itemize}
	\item \textit{Página 39, fórmula (3.23):} $$\frac{dGli_3}{dt}=\frac{r_{g3b}}{\textbf{Gli}}\dots$$ En referencia a el uso de Gli como factor representativo del grado de activación del sistema frente a $Ptc$.
	\end{itemize}
\end{document}
