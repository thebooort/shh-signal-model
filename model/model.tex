\chapter{Modelado teórico}

\label{ch:model}

\section{Consideraciones previas}


\section{Modelo final}

La mayoría de cuentas del apartado se han generado con la ayuda de \cite{sympy}.

\begin{equation}
\frac{dGli}{dt} = BEWARE(Gli, Gli_3, Gli3R)-k_{deg}Gli.
\end{equation}

\begin{equation}
\frac{dGli_3}{dt} = \frac{r_{g3b}}{Ptc}-Gli_3\left(k_{deg}+\frac{k_{g3rc}}{K_{g3rc}+Signal}\right).
\end{equation}

\begin{equation}
\frac{dGli3R}{dt}= Gli_3\left(\frac{k_{g3rc}}{K_{g3rc}+Signal}\right)-k_{deg}Gli3R.
\end{equation}

\begin{equation}
\frac{dPtc}{dt} = BEWARE(Gli, Gli_3, Gli3R)-k_{degp}Ptc.
\end{equation}


Donde tenemos, por definición:
 \begin{equation}
Signal=\frac{\frac{Shh}{k_{shh}} + 1}{\frac{Shh}{k_{shh}} + 1 + \frac{Ptc}{k_{ptc}}},
 \end{equation}

y,


\begin{equation}
BEWARE(Gli, Gli_3, Gli3R)=\frac{c_{b}}{1 + \frac{k_{RNAP}}{F_{reg}(Gli, Gli_3, Gli3R) RNAP}},
\end{equation}

donde solo nos queda describir $F_{reg}$. En el caso de de gradientes opuestos y no/total cooperatividad de los factores de transcripción nos queda:

\begin{equation}
F_{reg}=\frac{1 + \frac{1}{c} \left(\frac{Gli a_{Gli}}{k_{Gli}} c + \frac{Gli_{3} a_{Gli3}}{k_{Gli3R}} c + \frac{Gli3R c}{k_{Gli3R}} r_{Gli3R} + 1\right)^{3} - \frac{1}{c}}{1 + \frac{1}{c} \left(\frac{Gli c}{k_{Gli}} + \frac{Gli_{3} c}{k_{Gli3R}} + \frac{Gli3R c}{k_{Gli3R}} + 1\right)^{3} - \frac{1}{c}}
\end{equation}






Podemos desarrollar las funciones en cada uno de los términos, quedándonos las siguientes expresiones:
\begin{equation}
\frac{dGli}{dt}=- Gli k_{deg} + \frac{c_{b}}{1 + \frac{k_{RNAP} \left(1 + \frac{1}{c} \left(\frac{Gli c}{k_{Gli}} + \frac{Gli_{3} c}{k_{Gli3R}} + \frac{Gli3R c}{k_{Gli3R}} + 1\right)^{3} - \frac{1}{c}\right)}{RNAP \left(1 + \frac{1}{c} \left(\frac{Gli a_{Gli}}{k_{Gli}} c + \frac{Gli_{3} a_{Gli3}}{k_{Gli3R}} c + \frac{Gli3R c}{k_{Gli3R}} r_{Gli3R} + 1\right)^{3} - \frac{1}{c}\right)}}.
\end{equation}


\begin{equation}
\frac{dGli_3}{dt}=- Gli_{3} \left(k_{deg} + \frac{k_{g3rc}}{K_{g3rc} + \frac{\frac{Shh}{k_{shh}} + 1}{\frac{Shh}{k_{shh}} + 1 + \frac{ptc}{k_{ptc}}}}\right) + \frac{r_{g3b}}{ptc}.
\end{equation}

\begin{equation}
\frac{dGli3R}{dt}=Gli_{3} \left(- Gli3R k_{deg} + \frac{k_{g3rc}}{K_{g3rc} + \frac{\frac{Shh}{k_{shh}} + 1}{\frac{Shh}{k_{shh}} + 1 + \frac{ptc}{k_{ptc}}}}\right).
\end{equation}

\begin{equation}
\frac{dPtc}{dt}=\frac{c_{b}}{1 + \frac{k_{RNAP} \left(1 + \frac{1}{c} \left(\frac{Gli c}{k_{Gli}} + \frac{Gli_{3} c}{k_{Gli3R}} + \frac{Gli3R c}{k_{Gli3R}} + 1\right)^{3} - \frac{1}{c}\right)}{RNAP \left(1 + \frac{1}{c} \left(\frac{Gli a_{Gli}}{k_{Gli}} c + \frac{Gli_{3} a_{Gli3}}{k_{Gli3R}} c + \frac{Gli3R c}{k_{Gli3R}} r_{Gli3R} + 1\right)^{3} - \frac{1}{c}\right)}} - k_{deg p} Ptc.
\end{equation}