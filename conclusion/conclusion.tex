
\chapter{Conclusiones}

\label{ch:conclusions}

\section{Conclusiones y trabajo futuro}
\begin{enumerate}
	\item Hemos presentado en nuestro trabajo un estudio de dos modelos utilizados para entender el sistema de señalización de Shh. 
	\item  Dentro del modelo clásico hemos observado comportamientos cualitativos y como su origen puede residir también en otros parámetros no contemplados en el artículo original.
	\item  A su vez, hemos presentado un nuevo modelado para este problema siguiendo una estrategia distinta a la original.
	\item Este modelo ha demostrado comportarse de forma parecida al original, pero con sutiles diferencia en cuanto a la estabilidad. En concreto este modelo no presenta un swicth biestable en su comportamiento. Este hecho apoya la teoría de que este modelo capta de forma correcta la realidad biológica observada en la práctica.
	\item Nos gustaría seguir desarrollando este proyecto. Si bien el numérico a demostrado ser una interesante herramienta para captar tendencias cualitativas en nuestro modelo, necesitamos una solida prueba teórica.
	\item Para ello, a parte de utilizar barridos de parámetros mayores, queremos investigar si podemos obtener una prueba teórica de este resultado. Iniciamos este trabajo con un fin teórico, sin embargo la complejidad de las operaciones resiste nuestros envites de resolución actuales. Aún así, las pruebas numéricas sostienen que vamos por buen camino.
\end{enumerate}



 

 
 
 