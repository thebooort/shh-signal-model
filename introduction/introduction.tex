
\chapter{Introducción}


\label{ch:Introducción}

\section{El estudio cualitativo de sistemas con origen biológico: Motivación y objetivos}
Casi todas las ramas de las matemáticas poseen algún tipo de aplicación en el estudio de la biología. Debido a esto, el estudio de las herramientas utilizadas en biomatemática es increíblemente extenso, haciendo imposible un correcto análisis en pocas páginas. Por este simple hecho hemos decidido centrar este trabajo en el análisis teórico y numérico de dos cuerpos teóricos que, sin bien pueden parecer pequeños en comparación con el conjunto total, nos sirven para analizar multitud de sistemas con origen biológico.\\ 

Nos referimos, en este caso, al análisis de sistemas dinámicos a través del estudio de sus bifurcaciones y de la existencia de ciclos limite pues responde a unas situaciones que, si entendemos el sentido matemático de estos términos, se nos antojan puramente biológicas. Me refiero a, por una parte, el estudio de un sistema  biológico cuando alguno de los parámetros que en él aparece es perturbado o modificado, algo común en la experimentación biológica y que es de suma importancia conocer y predecir. 

Por otra parte la existencia de ciclos límite está íntimamente relacionada con comportamientos no solo estables o inestables si no, y lo que es( según la situación) más importante, la existencia de soluciones periódicas, un hecho que abundan en la naturaleza desde ciclos celulares hasta nuestro propio biorritmo circadiano.


Es de suponer por todo lector, que estudiar analíticamente cualquier sistema biológico, en la mayoría de los casos una tarea bastante ardua. Por eso mismo surge inevitablemente el estudio de los comportamientos no sólo anteriormente descritos si no, de todos los que de alguna manera se relacionan con esta descripción cualitativa que buscamos a través del análisis numérico.

 En este trabajo se ahonda en la estructura básica que debería tener un método que nos ayude a la construcción de diagramas de bifurcaciones (aunque también haremos breves comentarios sobre diagramas de fases). La mayoría de diagramas expuestos han sido realizados con los programas que se presentan en el apartado de software, a su vez presentaremos un caso final con todas las especificaciones posibles para que sirva de ejemplo del planteamiento, entrada y salida de datos.
 
 
Durante todo el trabajo iremos estudiando distintos casos de origen biológico, abarcando desde dinámicas celulares hasta dinámicas de ecología, que sirven de ejemplo de la teoría en la que he profundizado y sobre la que hemos dado no mas que pequeñas pinceladas.