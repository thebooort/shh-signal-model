\chapter{Análisis cualitativo}

\label{ch:simulation}


\section{Notas provisionales}
\begin{itemize}
	\item Por ahora apenas hay sensibilidad del modelo ha la variacion de la cantidad de Shh
	\item Se observa que la intensidad de la activacion transcripcional ($a_{Gli,Gli3}$) si que provoca cambios significativos en el comportamiento cualitativo
	\item la cantidad de ARNP tambien tiene un impacto importante. Su disminucion provoca un cambio disminutivo en el punto fijo
	\item cambio en el comportamiento del crecimiento de Gli en las cercanias de 0.07 ARNP
	\item el mayor cambio se observa al modificar la constante del beware. Sobre 0.27 obtenemos un estado estacionario de Gli como el que se obtiene en \cite{schaffer}
\end{itemize}



\section{Parámetros}
\begin{table}[h]
\begin{center}

\begin{tabular}{ |p{3cm}||c|p{3cm}|p{3cm}|  }
	\hline
	\multicolumn{4}{|c|}{Tabla de parámetros, operador \textit{BEWARE}} \\
	\hline
	Parámetro & Valor & Descripción & Fuente\\
	\hline
	$c $  & 1    &\tiny{Constante positiva (valor 1 implica cooperatividad total)} &   \cite{cambon1}\\
	$a_{Gli}$ &   4.35  & \tiny{Intensidad de represion transcripcional de Gli}   & \cite{cambon1}\\
	$a_{Gli3} $ & $4.35$ & \tiny{ Intensidad de represion transcripcional de Gli3 } &  \cite{cambon1}\\
	$r_{Gli3R}$   &$5\times10^{-5}$ & \tiny{ Intensidad de represion transcripcional de Gli } &  \cite{cambon1}\\
	$k_{Gli}$ &  $9\times10^{1}$  & \tiny{ Constante de disociacion de los activadores para los potenciadores geneticos } & \cite{cambon1}\\
	$k_{Gli3}$ & $9\times10^{1}$  & \tiny{ Constante de disociacion de los activadores para los potenciadores geneticos }   & \cite{cambon1}\\
	$k_{Gli3R}$ & $9\times10^{1}$ & \tiny{ Constante de disociacion de los represores para los potenciadores geneticos }   & \cite{cambon1}\\
	
	$k_{RNAP}$& 1  &  \tiny{Afinidad de unión   de RNA polimerasa} & \cite{cambon1}\\
	$RNAP$& 1  & \tiny{Concentración de RNA polimerasa} & \cite{cambon1}\\
	$c_b$& 1 $ nMmin^{-1}$  & \tiny{ Constante del operador} & \cite{cambon1}\\
	\hline
\end{tabular}

\end{center}
\caption{Tabla de parámetros, operador \textit{BEWARE}}\label{beware_params}
\end{table}

\begin{table}[h]
\begin{center}
	
	\begin{tabular}{ |p{3cm}||c|p{3cm}|p{3cm}|  }
		\hline
		\multicolumn{4}{|c|}{Tabla de parámetros de \cite{schaffer}  } \\
		\hline
		Parámetro & Valor & Descripción & Fuente\\
		\hline
		$Shh $  & $0-30$    &\tiny{Cantidad de Shh} &   \cite{cambon1}\\
		$k_{Shh}$ &  $ 0.58-2.0nM$  & \tiny{Constante de disociación de los enlaces Ptc-Shh}   & \cite{cambon1}\\
		$k_{Ptc} $ & $8.3\times10^{-11}M$ & \tiny{ Mitad de la máxima concentración de Ptc que inhibe la señal de Smo } &  \cite{cambon1}\\
		$k_{deg}$   &$0.009min^{-1} $ & \tiny{ Constante de degradacion de todas las moleculas Gli } &  \cite{cambon1}\\
		
		$k_{g3rc}$ &  $0.012min^{-1}$  & \tiny{ Constante deconversion de Gli3 en Gli3R} & \cite{schaffer}\\
		$r_{g3b}$ & $1.6\times10^{-19}M^2/min$  & \tiny{ Tasa basal de sintesis de Gli3 }   & \cite{schaffer}\\
		$K_{g3rc}$ & $0.1$ & \tiny{ Constante de sensibilidad de la conversioon a fuerza de la señal }   & \cite{schaffer}\\
		
		$k_{deg_p}$& $0.09min^{-1} $ &  \tiny{constante de degradacion de Ptc} & \cite{cambon1}\\
		
		\hline
	\end{tabular}
	
\end{center}
\caption{Tabla de parámetros de \cite{schaffer} }\label{param_2}
\end{table}